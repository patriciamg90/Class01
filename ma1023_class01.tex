
\begin{frame}
\begin{center}
\underline{Lecture 1: The L'Hopital's Rule}	
\end{center}		
Contents
\begin{itemize}
	\item   The Quotients rule for limits. 
	\item  Indeterminate forms \[ 0/0,\, \infty/ \infty \]
	
	and L'Hospital rule  
\end{itemize}



Objectives 
\begin{itemize}
	\item  Identify and apply intermediate forms and L'Hospital rule. 
	\item Distinguish the Quotients rule for limits and L'Hospital rule. 
\end{itemize}


Materials in the textbook:  Section 4.5 (Examples 1-3) 	
\end{frame}
%%%%%%%%%%%%%%%%%%%%%%%%%%%%%%%%%%%%%%%%%%%%%%%%%	

\begin{frame}
Recall:

\begin{mybox}{Theorem: The quotient rule for limits}{}
	If $L,\,M$ and $c$  are real numbers and $\lim_{x \rightarrow c} f(x)=L$ and $\lim_{x \rightarrow c} g(x)=M$, then 
	\begin{equation*}
		\lim_{x \rightarrow c} \dfrac{f(x)}{g(x)}=\dfrac{L}{M},	
	\end{equation*}
	provided $M \neq 0.$
	
\end{mybox}	
	
\begin{myexample}{}{}
Find the limit of   $\displaystyle \lim_{x \to 0} \frac{\sin(x)}{x+1}$.	
	
\end{myexample}
		
	
	
\end{frame}


\begin{frame}

\begin{myexample}{}{}
	Find $\lim_{x \rightarrow 1} \dfrac{3x-3}{3x-1}.$	
\end{myexample}
		
	
\end{frame}
%%%%%%%%%%%%%%%%%%%%%%%%%

\begin{frame}{L'Hospital's rule}
\begin{mybox}{L'Hospital's rule}
For functions $f$ and $g$ which are differentiable on an open interval I except possibly at a point $c$ contained in $I$, if

\begin{itemize}
	\item ${\displaystyle \lim _{x\to a}f(x)=\lim _{x\to a}g(x)=0{\text{ or }}\pm \infty ,}  $ 
	\item $  {\displaystyle g'(x)\neq 0}$ for all $x\in I$  with $x\neq  a$, and
	\item $  \displaystyle  \lim _{x\to a}{\frac {f'(x)}{g'(x)} } $ 
	exists, 	
\end{itemize}
then \qquad 
$ \displaystyle \lim _{x\to a}{\frac {f(x)}{g(x)}}=\lim _{x\to a}{\frac {f'(x)}{g'(x)}}.$	
\end{mybox}
\end{frame}

%%%%%%%%%%%%
\begin{frame}
Now applying  L'Hospital's rule. 

%\begin{exm}
%		Find the limit of   $\displaystyle \lim_{x \to 0} \frac{2(1-\cos(x))}{x^2+2x}$.		
%\end{exm}
\begin{myexample}{}{}
Find $\lim_{x\rightarrow \infty} \dfrac{4x^3-6x^2+1}{2x^3 -10x +3}$	
\end{myexample}	
	
	
\end{frame}

\begin{frame}
\begin{myexample}{}{}
Find the limit of   $\displaystyle \lim_{x \to 0} \frac{2(1-\cos(x))}{x^2}$.	
\end{myexample}

		
	
\end{frame}


\begin{frame}
\begin{myexample}{}{}
	Find the limit of   $\displaystyle \lim_{x \to 0} \frac{\sin(x)}{x^2}$.		
\end{myexample} 
\end{frame}

%%%%%%%%%%%%%%%%%%%%%%%%%%%%
\begin{frame}
	
L'Hospital's rule  uses derivatives to help evaluate limits involving indeterminate forms:
\[0/0 \quad \infty/\infty,  \quad  0\times \infty, \quad
\infty- \infty, \quad 0^0,\quad 1^\infty \text{ and } \infty^0.\]


\begin{mybox}{Remark}{}
	The expression $1/0$ is not commonly regarded as an indeterminate form.
\end{mybox}	
	
\end{frame}

\begin{frame}{Self-quiz problems}
\begin{enumerate}	

\item<handout:0>	Find the limit of   $\displaystyle \lim_{x \to 1} \frac{\sin(x)}{x}$.




\item<handout:0>	Describe l'Hospital’s Rule. How do you know when to use the rule
	and when to stop? Give an example.

\end{enumerate}
	
\end{frame}
	
